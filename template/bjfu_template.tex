\documentclass{bjfu}
%%%%%%%%%%%%%%%%%%%%%%%%%%%%%%%%%%%%%%%%%%%%%%%%%%%
\title{北京林业大学本科毕业论文模板}
\author{作者}
%\date{\today}
% 汉语标题
\bjfuTitle{北京林业大学本科毕业论文模板}
% 英语标题
\bjfuTitleEn{An Bachelor Thesis Template of Beijing Forestry University}
% 汉语作者
\bjfuAuthor{作者1}
% 英语作者
\bjfuAuthorEn{Author}
% 汉语指导教师
\bjfuSupervisor{导师}
\bjfuSupervisorTitle{教授}
% 英语指导教师
\bjfuSupervisorEn{Teacher}
% 汉语班级
\bjfuClass{班级}
% 英语班级
\bjfuClassEn{Class}
% 学院
\bjfuCollege{学院}
% 专业
\bjfuMajor{专业}
% 学号
\bjfuId{110614309}

% 汉语摘要
\bjfuAbstract{
根据《北京林业大学论文撰写规范及模板》, 本科生毕业论文必须符合相关规定。 一方面, 这些规定不尽不实, 存在着相当多的错漏; 另一方面, 即将毕业的学生实在很难静下来调整论文格式。 因此, 我们一起构建了本模板。
}

% 英语摘要
\bjfuAbstractEn{
In Beijing Forestry University, bachelor thesis have to meet some requirements. But relevant regulation is not consistency and complete. And it's hard for undergraduate to calm down and proofread their newly completed work. So, we construct this template in order to reduce their cost.
}

% 汉语关键字
\bjfuKeywords{毕业论文, 论文模板, 北京林业大学, 本科生}

% 英语关键字
\bjfuKeywordsEn{Thesis, Template, Beijing Forestry University, Bachelor}
%%%%%%%%%%%%%%%%%%%%%%%%%%%%%%%%%%%%%%%%%%%%%%%%%%%
\begin{document}
\makeBjfuTitlePage

\section{快速开始}
快速开始 \cite{刘海洋2013LATEX}

内容内容内容内容内容内容内容内容内容内容内容内容内容内容内容内容内容内容内容内容内容内容内容内容内容内容内容

\subsection{快速}

内容内容内容内容内容内容内容内容内容内容内容内容内容内容内容内容内容内容内容内容内容内容内容内容内容内容内容

这是一个有序列表

\begin{enumerate}[labelsep = .5em, leftmargin = 0pt, itemindent = 3em]
    \item 这是一个行内公式 $y = ax^2+b$
    
    \item This is the second item %小标题
    
    % 段落内容
    \setlength{\parindent}{2em}内容内容内容内容内容内容内容内容内容内容内容内容内容内容内容内容内容内容内容内容内容内容内容内容内这容内容内容内容内容内容内容内容内容内容内容内容内容内容内容内容内容内容内容内容内容内容内容内容内容内容内容内容内容内容内容
    \item This is the third item 
    
    % 子列表(不需要可删除)
    \begin{enumerate}[itemindent=2em]
        \item This is the first item
        \item This is the second item
        \item This is the third item
    \end{enumerate}
\end{enumerate}


这是一个行内有序列表
\begin{enumerate}[fullwidth,itemindent=2em]
    \item This is the first item
    \item This is the second item 内容内容内容内容内容内容内容内容内容内容内容内容内容内容内容内容内容内容内容内容内容内容内容内容内这容内容内容内容内容内容内容内容内容内容内容内容内容内容内容内容内容内容内容内容内容内容内容内容内容内容内容内容内容内容内容
    
    这是一个单行有编号公式
    \begin{equation}
    y=ax^2+b
    \end{equation}
    \item This is the third item 
\end{enumerate}

\subsubsection{快速快速开始}

内容内容内容内容内容内容内容内容内容内容内容内容内容内容内容内容内容内容内容内容内容内容内容内容内容内容内容内容内容内容内容内容内容内容内容内容内容内容内容内容内容内容内容内容内容内容内容内容内容内容内容内容内容内容
\begin{equation}
    y=vt^2
\end{equation}


这是一个无序列表
\begin{itemize}
    \item Fedora Versions
    \begin{itemize}
        \item Fedora 8
        \item Fedora 9
        \begin{itemize}
            \item Werewolf
            \item Sulphur
            \begin{itemize}
                \item 2007-05-31
                \item 2008-05-13
            \end{itemize}
        \end{itemize}
    \end{itemize}
    \item Fedora Spin
    \item Fedora Silverblue
\end{itemize}

图片的插入方法的界如图\ref{Fig:test}所示
\begin{figure}[htbp] %H为当前位置,!htb为忽略美学标准,htbp为浮动图形
    \centering %图片居中
    \includegraphics[width=0.3\textwidth]{images/test.png} %插入图片,[]中设置图片大小,{}中是图片文件名
    \bjfuFigureCaption{测试图片}{Test Picture} % 第一个参数为中文名,第二个参数为英文名
    \label{Fig:test} %用于文内引用的标签
\end{figure}

内容内容内容内容内容内容内容内容内容内容内容内容内容内容内容内容内容内容内容内容内容内容内容内容内容内容内容内容内容内容内容内容内容内容内容内容内容内容内容内容内容内容内容内容内容内容内容内容内容内容内容内容内容内容

表格的插入方法的界如表\ref{Table:test}所示
\begin{table}[h]  %table 里面也可以嵌套tabular,只有tabular是不能加标题的
    \bjfuTableCaption{测试表格}{Test Table}
    \renewcommand\arraystretch{1.5}
    \centering  %表格居中
    \setlength{\tabcolsep}{7mm}{
        \begin{threeparttable}
        \begin{tabular}{c|c|c|c}  %按列居中对齐,并在中间划线 左右中分别用字母l,r,c表示
            \hline
            列名\tnote{1} & 类型\tnote{2} & 默认值 & 注释 \\ [0.5ex]  %增加行宽
            \hline
            id & int & Auto Inc &  用户id,主键 \\
            \hline
            name & text &  & 昵称 \\
            \hline
            phone & char(11) & - & 手机号 \\
            \hline
            avatar & text & - & 头像地址 \\
            \hline
        \end{tabular}
        %% 注释开始
        \begin{tablenotes}
            \tiny
            \item 无编号注释。     
            \item[1] 列名的注释。 %根据编号使用\tnote{编号}引用
            \item[2] 表名的注释。
        \end{tablenotes}       
        %% 注释结束    
        \end{threeparttable}
    }
    \label{Table:test} %用于文内引用的标签
\end{table}

\subsection{代码}

\begin{minted}[breaklines,baselinestretch=1]{python}
if __name__ == '__main__':
    print('hello world')
\end{minted}

\bjfuack

\subsection{作者简介}
\subsubsection{王政}

\subsubsection{王凌霄}

马起园先生作为ApTex的作者在2015年本模板的早期编写阶段给出了宝贵建议, 给予了极大的帮助。


%用biblatex来管理参考文献
\bjfubib{reference}

\end{document}
