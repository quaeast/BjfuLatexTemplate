\documentclass{bjfu}

%%% 论文基础配置

% 标题 {中文}{英文}
\bjfuTitle {北京林业大学本科毕业论文模板}
{An Bachelor Thesis Template of Beijing Forestry University}
% 作者 {中文}{英文}
\bjfuAuthor{作者}{Author}
% 指导教师 {中文}{英文}
\bjfuSupervisor{导师}{Teacher}
% 指导教师职称
\bjfuSupervisorTitle{教授}
% 班级 {中文}{英文}
\bjfuClass{班级}{Class}
% 专业
\bjfuMajor{专业}
% 学院
\bjfuCollege{XX学院}
% 学号
\bjfuId{XXXXXXXXX}

%%% 摘要与关键字

% 汉语摘要
\bjfuAbstract{
根据《北京林业大学论文撰写规范及模板》, 本科生毕业论文必须符合相关规定。 一方面, 这些规定不尽不实, 存在着相当多的错漏; 另一方面, 即将毕业的学生实在很难静下来调整论文格式。 因此, 我们一起构建了本模板。
}

% 英语摘要
\bjfuAbstractEn{
In Beijing Forestry University, bachelor thesis have to meet some requirements. But relevant regulation is not consistency and complete. And it's hard for undergraduate to calm down and proofread their newly completed work. So, we construct this template in order to reduce their cost.
}

% 汉语关键字
\bjfuKeywords{毕业论文, 论文模板, 北京林业大学, 本科生}

% 英语关键字
\bjfuKeywordsEn{Thesis, Template, Beijing Forestry University, Bachelor}

\begin{document}
\makeBjfuTitlePage

%%% 正文

\section{这是一级标题}
\subsection{这是二级标题}
\subsubsection{这是三级标题}
这是正文内容这是正文内容这是正文内容这是正文内容这是正文内容这是正文内容这是正文内容这是正文内容这是正文内容这是正文内容这是正文内容这是正文内容这是正文内容这是正文内容这是正文内容这是正文内容这是正文内容这是正文内容这是正文内容这是正文内容这是正文内容这是正文内容这是正文内容这是正文内容这是正文内容这是正文内容这是正文内容这是正文内容这是正文内容这是正文内容这是正文内容这是正文内容这是正文内容这是正文内容这是正文内容这是正文内容这是正文内容这是正文内容这是正文内容这是正文内容这是正文内容这是正文内容这是正文内容这是正文内容这是正文内容这是正文内容这是正文内容这是正文内容这是正文内容这是正文内容

这是一个在正文中插入图片的例子,插入效果如图\ref{Fig:test}所示
\begin{figure}[htbp] %H为当前位置,!htb为忽略美学标准,htbp为浮动图形
    \centering %图片居中
    \includegraphics[width=0.3\textwidth]{images/test.png} %插入图片,[]中设置图片大小,{}中是图片文件名
    \bjfuFigureCaption{测试图片}{Test Picture} % 第一个参数为中文名,第二个参数为英文名
    \label{Fig:test} %用于文内引用的标签
\end{figure}

这是正文内容这是正文内容这是正文内容这是正文内容这是正文内容这是正文内容这是正文内容这是正文内容这是正文内容这是正文内容这是正文内容这是正文内容这是正文内容这是正文内容这是正文内容这是正文内容这是正文内容这是正文内容这是正文内容这是正文内容这是正文内容这是正文内容这是正文内容这是正文内容这是正文内容这是正文内容这是正文内容这是正文内容这是正文内容这是正文内容这是正文内容这是正文内容这是正文内容这是正文内容这是正文内容这是正文内容这是正文内容这是正文内容这是正文内容这是正文内容这是正文内容这是正文内容这是正文内容这是正文内容这是正文内容这是正文内容这是正文内容这是正文内容这是正文内容这是正文内容这是正文内容这是正文内容这是正文内容这是正文内容这是正文内容这是正文内容这是正文内容

\section{使用教程}
\subsection{参考文献的配置}
这是一个书籍引用\cite{引用1},这是一个文献引用\cite{引用2}。参考文献的配置在reference.bib中,参考文献的配置使用BibTex,你可以自行搜索BibTex的使用方法。

\subsection{参考文献的分类与表格的使用}
你可以参考文献的类别如表\ref{Table:reference}所示:
\begin{table}[h] %table 里面也可以嵌套tabular,只有tabular是不能加标题的
    \bjfuTableCaption{BibTex参考文献配置}{BibTex reference arrangement}
    \footnotesize %设置为小五号字体
    \renewcommand\arraystretch{1.5}
    \centering %表格居中
    \setlength{\tabcolsep}{7mm}{
        \begin{threeparttable}
            \begin{tabular}{c|c} %按列居中对齐,并在中间划线 左右中分别用字母l,r,c表示
                \hline
                配置项\tnote{1} & 含义 \\
                \hline
                @article & 一篇来自期刊或杂志的文章 \\
                \hline
                @book & 一本有明确出版商的书\\
                \hline
                @booklet & 已印刷和装订的作品,但没有指定的出版商或赞助机构  \\
                \hline
                @inproceedings & 会议记录中的一篇文章 \\
                \hline
                @conference & 与inproceedings相同 \\
                \hline
                @inbook& 一本书的一部分,可以是一章(或一节或其他)和/或一系列的页 \\
                \hline
                @incollection &一本书的一部分,有自己的标题 \\
                \hline
                @manual &技术文件 \\
                \hline
                @mastersthesis &一篇硕士论文 \\
                \hline
                @phdthesis& 一篇博士论文\\
                \hline
                @misc& 在没有其他适合的情况下使用这种类型 \\
                \hline
                @proceedings &一个会议的记录 \\
                \hline
                @techreport &由学校或其他机构出版的报告,通常在一个系列中编号。 \\
                \hline
                @unpublished& 有作者和标题,但没有正式出版的文件 \\
                \hline
                @collection& 不是一个标准的条目类型。使用proceedings代替。 \\
                \hline
                @patent& 不是一个标准的条目类型。 \\
                \hline
            \end{tabular}
            %% 注释开始
            \begin{tablenotes}
                \tiny
                \item  这是一个表格注释
                \item[1]  这是一个带编号表格注释%根据编号使用\tnote{编号}引用
            \end{tablenotes} 
            %% 注释结束 
        \end{threeparttable}
    }
    \label{Table:reference} %用于文内引用的标签
\end{table}

\subsection{参考文献的配置项与有序列表的使用}

不同的参考文献含有不同的参数列表,具体的参数列表如下所示:
\begin{enumerate}[labelsep = .5em, leftmargin = 0pt, itemindent = 3em]
    \item  @article
    期刊杂志的论文
    
    \setlength{\parindent}{2em}必要域: author, title, journal, year.
    可选域: volume, number, pages, month, note.
    
    \item @book
    公开出版的图书
    
    \setlength{\parindent}{2em}必要域: author/editor, title, publisher, year.
    可选域: volume/number, series, address, edition, month, note.
    
    \item @booklet
    无出版商或作者的图书
    
    \setlength{\parindent}{2em}必要域: title.
    可选域: author, howpublished, address, month, year, note.
    
    \item @conference
    等价于 inproceedings
    
    \setlength{\parindent}{2em}必要域: author, title, booktitle, year.
    可选域: editor, volume/number, series, pages, address, month, organization, publisher, note.
    
    \item @inbook
    书籍的一部分章节
    
    \setlength{\parindent}{2em}必要域: author/editor, title, chapter and/or pages, publisher, year.
    可选域: volume/number, series, type, address, edition, month, note.
    
    \item @incollection
    书籍中带独立标题的章节
    
    \setlength{\parindent}{2em}必要域: author, title, booktitle, publisher, year.
    可选域: editor, volume/number, series, type, chapter, pages, address, edition, month, note.
    
    \item @inproceedings
    会议论文集中的一篇
    
    \setlength{\parindent}{2em}必要域: author, title, booktitle, year.
    可选域: editor, volume/number, series, pages, address, month, organization, publisher, note.
    
    \item @manual
    技术文档
    
    \setlength{\parindent}{2em}必要域: title.
    可选域: author, organization, address, edition, month, year, note.
    
    \item @mastersthesis
    硕士论文
    
    \setlength{\parindent}{2em}必要域: author, title, school, year.
    可选域: type, address, month, note.
    
    \item @misc
    其他
    
    \setlength{\parindent}{2em}必要域: none
    可选域: author, title, howpublished, month, year, note.
    
    \item @phdthesis
    博士论文
    
    \setlength{\parindent}{2em}必要域: author, title, year, school.
    可选域: address, month, keywords, note.
    
    \item @proceedings
    会议论文集
    
    \setlength{\parindent}{2em}必要域: title, year.
    可选域: editor, volume/number, series, address, month, organization, publisher, note.
    
    \item@techreport
    教育,商业机构的技术报告
    
    \setlength{\parindent}{2em}必要域: author, title, institution, year.
    可选域: type, number, address, month, note.
    
    \item @unpublished
    未出版的论文,图书
    
    \setlength{\parindent}{2em}必要域: author, title, note.
    可选域: month, year.
\end{enumerate}


\subsection{行内有序列表与子列表的使用}
上一接我们简单了解了如何创建一个有序列表,接着我们来学习如何创建一个含有子列表的有序列表。
\begin{enumerate}[labelsep = .5em, leftmargin = 0pt, itemindent = 3em]
    \item 这是小标题 % 小标题
    % 段落内容
    \setlength{\parindent}{2em}这是列表内容

    \item 这是小标题 % 小标题
    % 段落内容
    \setlength{\parindent}{2em}这是列表内容

    \item 这是小标题 % 小标题
    % 段落内容
    \setlength{\parindent}{2em}这是列表内容
    % 子列表(不需要可删除)
    \begin{enumerate}[itemindent=2em]
        \item 这是子列表项 
        \item 这是子列表项 
        \item 这是子列表项 
    \end{enumerate}
\end{enumerate}
有时列表并不是如上的小标题格式,此时我们需要行内有序列表。
\begin{enumerate}[fullwidth,itemindent=2em]
    \item 这是一个行内有序列表
    \item 列表标号的缩进值与段落内容一直
    \begin{equation}
        y=ax^2+b
    \end{equation}
    \item This is the third item 
\end{enumerate}


\subsection{无序列表与公式插入}
这是一个无需列表的示例,与有序列表一样无需列表也可以创建子列表
\begin{itemize}
    \item 这是一个列表项
    \item 这是一个列表项
    \item 公式插入
    \begin{itemize}
        \item 这是一个行内公式$y=ax^2+b$
        \item Fedora 9
        \begin{itemize}
            \item Werewolf
            \item Sulphur
            \begin{itemize}
                \item 2007-05-31
                \item 2008-05-13
            \end{itemize}
        \end{itemize}
    \end{itemize}
    
\end{itemize}


\begin{enumerate}[labelsep = .5em, leftmargin = 0pt, itemindent = 3em]
    \item 这是一个行内公式 $y = ax^2+b$
    
    \item This is the second item %小标题
    
    % 段落内容
    \setlength{\parindent}{2em}内容内容内容内容内容内容内容内容内容内容内容内容内容内容内容内容内容内容内容内容内容内容内容内容内这容内容内容内容内容内容内容内容内容内容内容内容内容内容内容内容内容内容内容内容内容内容内容内容内容内容内容内容内容内容内容
    \item This is the third item 
    
    % 子列表(不需要可删除)
    \begin{enumerate}[itemindent=2em]
        \item This is the first item
        \item This is the second item
        \item This is the third item
    \end{enumerate}
\end{enumerate}




\subsubsection{这是三级标题}

内容内容内容内容内容内容内容内容内容内容内容内容内容内容内容内容内容内容内容内容内容内容内容内容内容内容内容内容内容内容内容内容内容内容内容内容内容内容内容内容内容内容内容内容内容内容内容内容内容内容内容内容内容内容
\begin{equation}
    y=vt^2
\end{equation}




内容内容内容内容内容内容内容内容内容内容内容内容内容内容内容内容内容内容内容内容内容内容内容内容内容内容内容内容内容内容内容内容内容内容内容内容内容内容内容内容内容内容内容内容内容内容内容内容内容内容内容内容内容内容

\subsection{代码}

\begin{minted}[breaklines,baselinestretch=1]{python}
if __name__ == '__main__':
    print('hello world')
\end{minted}

\subsection{作者简介}
\subsubsection{测试}
内容


% 致谢
\bjfuThanks
这是致谢正文这是致谢正文这是致谢正文这是致谢正文这是致谢正文这是致谢正文这是致谢正文这是致谢正文这是致谢正文这是致谢正文这是致谢正文这是致谢正文这是致谢正文这是致谢正文这是致谢正文这是致谢正文这是致谢正文这是致谢正文这是致谢正文。

这是致谢正文这是致谢正文这是致谢正文这是致谢正文这是致谢正文这是致谢正文这是致谢正文这是致谢正文这是致谢正文这是致谢正文这是致谢正文这是致谢正文这是致谢正文这是致谢正文这是致谢正文这是致谢正文这是致谢正文这是致谢正文这是致谢正文这是致谢正文这是致谢正文这是致谢正文这是致谢正文这是致谢正文这是致谢正文这是致谢正文这是致谢正文。

这是致谢正文这是致谢正文这是致谢正文这是致谢正文这是致谢正文这是致谢正文这是致谢正文这是致谢正文这是致谢正文。



%参考文献 - 在reference.bib中编辑
\bjfubib{reference}

\end{document}
